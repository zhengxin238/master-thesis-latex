%%%
%%% @(#) tuc-thesis.tex -- Documentation for the TUC*-files
%%%
%%% Time-stamp:  <2019-11-17 13:43:53 tmjb>
%%%
%%%
%%%       @(#) $Id: tuc-thesis.tex,v 1.1 2016/11/29 08:21:44 tmjb Exp tmjb $
%%% @(#) $Keywords: TUC, Clausthal-University of Technology, Thesis Class, Thesis Style, TUC Style $
%%%
%%% 
%%%           File: /Users/tmjb/lib/texmf/tex/doc/latex/TUC/tuc-thesis.tex
%%%        Project: TUC
%%%    Description: This file documents and describes the TUC-thesis
%%%                 class and style files.
%%%        Version: $Revision: 1.1 $
%%%         Author: tmjb -- Jan Braun <braun@rz.tu-clausthal.de>
%%%     Maintainer: tmjb -- Jan Braun <braun@rz.tu-clausthal.de>
%%%  Creation-Date: Wed Jun  1 2016 -- Jan Braun <braun@rz.tu-clausthal.de>
%%%      Copyright: (c) 2016 Jan Braun
%%%

%%% ------------------------------------------------------ &Change Log ---
%%%
%%% $Log: tuc-thesis.tex,v $
%%% Revision 1.1  2016/11/29 08:21:44  tmjb
%%% Initial revision
%%%


%%% ======================================================== &Preamble ===
\documentclass[
   a4paper,                      % use regular A4 sized paper.
   tucfont,                      % use the Stone fonts
   german,                       % use german Logo and section names
   % english,                      % use english logo instead
   tuctitle,                     % use the TUC title style
   % preprint,                     % use preprinted titlepage paper
   % 11pt,                         % main font size
   intoc,                        % for the nomenclature package
   twoside,                      % use both sides of the paper
   captions=tableheading,        % Captions on top of a table
   openright,                    % open new chapters on a right side
   % draft,                        % draft version, opposite of final
   final,                        % final and last version
   ]{tuc-thesis}                 % our own (meta) class


%%% -------------------------------------------------------- &Packages ---
\usepackage[main=ngerman,english]{babel} % new german orthography
\usepackage[utf8]{inputenc}      % Text is inserted in UTF-8
% \usepackage[T1]{fontenc}         % use modern Font encoding
\usepackage[hang,raggedright]{subfigure} % put more than one figure in a % figre
                                 % environment 
\usepackage{graphicx}            % insert graphics files and such
\usepackage{booktabs}            % for better lines in tables
\usepackage{array}               % for better tables
\usepackage{longtable}           % for tables, that span across
                                 % multiple pages
\usepackage{mflogo}              % for the metafont logo
\usepackage[newfloat=true]{minted} % for better listings?
\usepackage{weiterbildung}       % used for printing usage of LaTeX
                                 % commands and such 
\usepackage{nicefrac}            % for nicer frations like 1/2
\usepackage{units}               % for nice printing of units like N
                                 % or km/h
\usepackage{textcomp}            % for \textdegree to obtain °
\usepackage{nomencl}             % explain Fremdwort
\usepackage[style=numeric]{biblatex} % newer and more versatile
                                 % bibliography tool




%%% ------------------------------------------------------- &Variablen ---
%%% Some useful variables . . .
\TUCauthor{Dipl.-Ing.\ Jan Braun} % similar to \author
\TUCtitle{Die TU Clausthal Pakete} % similar to \title
\TUCsubtitle{Eine kurze Anleitung für Version 0.8}
                                 % similar to \subtitle (defined by
                                 % KOMA-Script classes
\TUCdate{17.\ November 2019}      % similar to \date
\TUCthesistype{Anleitung}        % usually Bachelorarbeit or
                                 % Master-Thesis
\TUCinstitute{Dezernat 4}
\Firstcensor{Dipl.-Ing.\ Jan Braun}
\Secondcensor{Vizepräsident Prof.\ Dr.-Ing.\ Gunther Brenner}
\Thirdcensor{Präsident Prof.\ Dr.\ Joachim Schachtner}

\hypersetup{%
  unicode=false,%
  pdftoolbar=true,%
  pdfmenubar=true,%
  pdffitwindow=false,%
  pdfstartview={FitV},%
  pdfpagelayout={TwoColumnLeft},%
  colorlinks=true,%
  linkcolor=TUCgreen,%
  citecolor=TUCred,%
  filecolor=TUCred,%
  urlcolor=TUCgreen,%
  pdfauthor={Dipl.-Ing. Jan Braun},
  pdftitle={Die TUC Pakete},
  pdfsubject={Eine kurze Anleitung zur Nutzung des Corporate Design
    der TU Clausthal in LaTeX},
  % pdfcreator={\LaTeX\ with package \flqq{}hyperref\frqq},
  % pdfproducer={pdfTeX \the\pdftexversion.\pdftexrevision},
  pdfkeywords={TUC, TUC Clausthal, Technische Universität Clausthal,
    Clausthal University of Technology, Corporate Design, Description
    for Usage in LaTeX},
  pdfnewwindow=true}

%% 
%% Some important settings for the presentation of listings
\setminted[latex]{frame=lines,linenos=true,fontsize=\small}


%% 
%% Correction to prevent printing the logo so close to the edge, that
%% the printer is not able to print it correctly.
% \setlength{\logovcorr}{15mm}
% \setlength{\logohcorr}{15mm}



%%% ----------------------------------------------------- &Definitions ---
%%% Some useful definitions . . .

%% \name will be used, to set names of persons in an unique manner.
\newcommand{\name}[1]{\textsc{#1}}%

%% Which bibliography database to use?
\bibliography{tuc-thesis}



%%%..................................................... &Nomenclature ...
%% 
%% use nomencl package to explain unknown wordings.
%% first: use the package.
\makenomenclature
%% To create this index, first run latex, than run
%% makeindex tuc-thesis.nlo -s nomencl.ist -o tuc-thesis.nls
%% before running latex again.
%% 
%% Use a more convienent section name
\renewcommand{\nomname}{Fremdwortverzeichnis}
%% Redefine the way, the entry is formatted.
\renewcommand{\nomlabel}[1]{\textsf{#1}\hfil}
\setlength{\nomlabelwidth}{2cm}
%% define a command, to insert with least efford, the word and its
%% description in one pass.  This command takes 2 (up to 3) arguments:
%%  1) the unknown word itself, which is to be put literally into the
%%     list of nomenclatures
%%  2) the explaination, which will be only visible in the list of
%%     nomenclatures
%%  3) Eventually the unknown word in a different type, to be output
%%     in the text instead of #1
\makeatletter
\newcommand{\fremdwort}[3][\@empty]{%
  \nomenclature{#2}{#3}%
  % If #1 is not set, we don't need it.
  \ifx\@empty#1\emph{#2}\else\emph{#1}\fi}%
\makeatother



%%% -------------------------------------------------------- &Includes ---
\includeonly{%
  zusammenfassung,
  tuc-cd,
  typographie,
  einleitung,
  latex-klassen,
  zusammenfassung,
}



%%% ======================================================== &Document ===
\begin{document}

%%% ---------------------------------------------------- &Front-Matter ---
%% Start with different page numbers
\frontmatter

%% First of all, beginn with a title page
\makeTUCtitle {committee election
}                  % similar to \title.

%% Give the assertion, that this file was created by me.
\TUCassertion

%% Lets have a short summary first.
\include{zusammenfassung}

%% Declare all kind of unknown words.
\printnomenclature{}



%% Now have the table of contents, list of figures, list of tables, etc.
\tableofcontents{}
\listoffigures{}
\listoftables{}
\lstlistoflistings{}



%%% ----------------------------------------------------- &Main-Matter ---
\mainmatter
\include{einleitung}
%%%
%%% @(#) tuc-cd.sty -- multipurpose LaTeX-style file
%%%
%%% Time-stamp:  <2019-11-20 17:17:28 tmjb>
%%%
%%%
%%% @(#) $Id: tuc-cd.sty,v 1.2 2016/07/05 16:05:48 tmjb Exp tmjb $
%%% @(#) $Keywords: TU Clausthal, TUC, Corporate Design, CD, Design-Vorlagen $
%%%
%%%
%%%          File: /home/tmjb/lib/temf/tex/latex/tuc-cd/tuc-cd.sty
%%%       Project: TUC-CD
%%%       Version: $Revision: 1.2 $
%%%   Description: Allgemeine LaTeX-Style-Datei mit
%%%                div. Vereinbarungen um verschiedene Design-Elemente
%%%                des neuen Corporate Designs der TU Clausthal
%%%                umzusetzen.  Die Datei ist bewusst allgemein
%%%                gehalten, um sie in verschiedenen Klassen und
%%%                Umgebungen (Brief, Aushang, Poster, ...) nutzen zu
%%%                können.  Falls Änderungen am CD gemacht werden,
%%%                müssen diese nur in diesem Style eingepflegt
%%%                werden, die anderen Klassen bleiben davon
%%%                unberührt.
%%%
%%%                Diese Datei beruht zu einem Teil auf Vorarbeiten
%%%                von Dominic Wäsch.
%%%        Author: tmjb -- Jan Braun <Jan.Braun@tu-clausthal.de>
%%%    Maintainer: tmjb -- Jan Braun <Jan.Braun@tu-clausthal.de>
%%% Creation-Date: 2007-03-08 -- Jan Braun <Jan.Braun@tu-clausthal.de>
%%%     Copyright: (c) 2007 Jan Braun
%%%

%%% ------------------------------------------------------ &Change Log ---
%%%
%%% $Log: tuc-cd.sty,v $
%%% Revision 1.3 2019/10/17 11:25 tmjb
%%% Summary: new length titleskip, better definition for new column
%%%          types L, C and R.
%%% 
%%% Revision 1.2  2016/07/05 16:05:48  tmjb
%%% Summary: Options for german, ngerman and english added.
%%%
%%% * tuc-cd.sty: Farbedefinitionen entfernt.
%%%
%%% * tuc-cd.sty: 1.1 added options for german, ngerman and english.
%%%
%%% Revision 1.1  2016/07/05 14:31:44  tmjb
%%% Initial revision


%%% ============================================================ &Code ===

%%% ---------------------------------------------------- &Package Name ---
% \NeedsTeXFormat{LaTeX2e}[1995/12/01]
\ProvidesPackage{tuc-cd}[2019/09/19 v0.1.3 TU Clausthal Coporate Design style definitions (JB)]


%%% ------------------------------------------------------- &Variables ---
%% We need some variables, to store and indicate which options (see
%% later) are choosen ...
\newif\if@draft                 % for Option draft
\newif\if@final                 % Oposite of draft
\newif\if@afourpaper            % papersize DIN A4
\newif\if@afivepaper            % papersize DIN A5
\newif\if@letter                % american letter paper size
\newif\if@watermark             % Using a watermark to clearly
                                % indicate draft status of work in
                                % progress.
\newif\if@beamer                % used for presentations
\newif\if@tucfont               % use the corporate design font
\newif\if@rgb                   % use colored items, logos, ...
\newif\if@cmyk                  % also color, but using another color
                                % scheme, suitable for printing shops
                                % ...
\newif\if@black                 % don't use colors at all.
\newif\if@english\@englishfalse % default for most texts
\newif\if@noindent              % Use gaps between paragraphs instead
                                % of indentations.


%%% --------------------------------------------------------- &Options ---
%% Options are definid by the main package or for this special style
%% file.  Just add them in square brackets like this:
%% \usepackage[final,foo,bar]{tuc-cd}
%% I advise, to avoid using final and draft in package related
%% options, as the are normally handed over from the main class
%% package options.


%% ............................................................ &Final ...
%% Option final is set, when you intend to print the final copy of
%% your work.
\DeclareOption{final}{%
  \@draftfalse%
  \@finaltrue%
  \PassOptionsToPackage{final}{graphicx}%
}%


%% ............................................................ &draft ...
%% Option draft is per default defined by the main class package,
%% unless you specify final.  This style defines it for continuity
%% reasons.  Draft and final are mutualy exclusive
\DeclareOption{draft}{%
  \@drafttrue%
  \@finalfalse%
  \@watermarkfalse%
  \PassOptionsToPackage{draft}{graphicx}%
}%


%% .......................................................... &A4paper ...
%% Option a4paper is set, to typeset on DIN A4 papersize.  It should
%% be the Default size.
\DeclareOption{a4paper}{%
  \@afourpapertrue%
  \@afivepaperfalse%
  \@letterfalse%
}%

%% .......................................................... &A5paper ...
%% Option a5paper is set, to typeset on DIN A4 papersize.
\DeclareOption{a5paper}{%
  \@afourpaperfalse%
  \@afivepapertrue%
  \@letterfalse%
}%


%% ........................................................... &letter ...
%% Option letterpaper is set, to typeset on american letter papersize.
\DeclareOption{letter}{%
  \@afourpaperfalse%
  \@afivepaperfalse%
  \@lettertrue%
}%


%% ........................................................ &Watermark ...
%% Print an watermark to all pages, to indicate work in progress.
%% This might sometimes be handy, but normaly it is very annoying, so
%% it is turned of by default.  If turned on, it will only appear if
%% draft mode is also turned on.  Final turnes off watermarking, of
%% course.
\DeclareOption{watermark}{%
  \if@draft%
  \@watermarktrue%
    \if@english%
      \PassOptionsToPackage{all,english,dvips}{draftcopy}%
    \else%
      \PassOptionsToPackage{all,german,dvips}{draftcopy}%
    \fi%
  \fi%
}%


%% ........................................................... &Beamer ...
%% 
%% This option is set, whenever someone wants to give a presentation.
\DeclareOption{beamer}{%
  \@beamertrue%
}%


%% .......................................................... &TUCFont ...
%% This option will use Stone fonts, as declared in the corporate
%% design manual.  If this option is not set, latin modern font will
%% be used instead.  This option will only set a variable.
\DeclareOption{tucfont}{%
  \@tucfonttrue%
}%
\DeclareOption{tucfonts}{%
  \@tucfonttrue%
}%


%% .............................................................. &RGB ...
%% Color, represented in RGB color modell is usefull for presentation
%% and while being viewed on monitors.  Printed color should be CMYK
%% color modell.
\DeclareOption{rgb}{%
  \@rgbtrue%
  \@cmykfalse%
  \@blackfalse%
  \PassOptionsToPackage{rgb}{tuc-colors}
}


%% ............................................................. &CMYK ...
%% This is the color modell used, when printing in 4 color printers
%% and professional print shops.
\DeclareOption{cmyk}{%
  \@rgbfalse%
  \@cmyktrue%
  \@blackfalse%
  \PassOptionsToPackage{cmyk}{tuc-colors}%
}%


%% ............................................................ &Black ...
%% No color at all.
\DeclareOption{black}{%
  \@rgbfalse%
  \@cmykfalse%
  \@blacktrue%
  \PassOptionsToPackage{black}{tuc-colors}%
}%


%% ........................................................... &German ...
\DeclareOption{german}{%
  \@englishfalse%
  \typeout{^^J tuc-cd.sty: 2) german language active ^^J }%
}%
\DeclareOption{ngerman}{%
  \@englishfalse%
  \typeout{^^J tuc-cd.sty: 3) german language active ^^J }%
}%


%% .......................................................... &English ...
%% English texts do use another logo and such.
\DeclareOption{english}{%
  \@englishtrue%
  \typeout{^^J tuc-cd.sty: 1) english language active ^^J }%
}%


%% ......................................................... &Noindent ...
%% Option noindent shall be used, when paragraphs shall be separated
%% by gaps instead of indenting them.
\DeclareOption{noindent}{%
  \@noindenttrue%
  \typeout{^^J tuc-cd.sty: noindent active ^^J }%
}%


%% ....................................................... &Undeclared ...
%% This one is for any option, which was not declared before.
\DeclareOption*{\OptionNotUsed}%
%% Set the options.
\ExecuteOptions{cmyk}%
\ProcessOptions\relax%



%%% -------------------------------------------------------- &Messages ---
%% Print out some warning messages, while compiling the text.
\if@draft
  \typeout{^^J tuc-cd.sty: WARNING! DRAFT-mode is in use!^^J }
\fi
\if@final
  \typeout{^^J tuc-cd.sty: FINAL-mode is in use!^^J }
\fi
\if@watermark
  \typeout{^^J tuc-cd.sty: WARNING! WATERMARK-mode is in use!^^J } 
\fi
\if@english
  \typeout{^^J tuc-cd.sty: 4) WARNING: english language is in use ^^J }%
\else%
  \typeout{^^J tuc-cd.sty: 4a) WARNING: german language is in use ^^J }%
\fi%
% \if@german
%   \typeout{^^J tuc-cd.sty: 5) WARNING: german language is in use ^^J }%
% \fi



%%% -------------------------------------------------------- &Packages ---
%%
%% To get all definitions of this style to work, we need some more
%% packages.  They are automatically loaded here.
%%
%% First, be sure to use the Stone fonts
\if@tucfont
  \RequirePackage{tuc-fonts}%
\fi
%% 
%% 
%% Define the colors
\RequirePackage{tuc-colors}
%% 
%% Required für SCRLAyers
\RequirePackage{scrlayer}
%% 
%% Here are some general packages.
\RequirePackage[gen,right]{eurosym} % Provides commands and
                                % fontdefinitions for the eurosymbol.
                                % It uses a generic Euro symbol of the
                                % actual font in use.  It is typeset
                                % right of the sum.
\RequirePackage{array}          % for better tables
\RequirePackage{booktabs}       % for better lines in tables
\RequirePackage{ragged2e} % new raggedright and raggedleft
                                % commands
%\RequirePackage{amssymb}        % for \Box-symbol used in Itemize



%%% ------------------------------------------------------ &Definitions ---

%% ......................................................... &Titlepage ...
%% 
%% The usual titlepage of an TUC document presents our logo on the top
%% left corner, as well as an grey box on the right hand side of the
%% paper.  The logo dimensions are given in th CD Manual: the square
%% area shall have 24 mm on a A4 paper.  Given the normal, german
%% logo, the height of the logo is also 24 mm.  In those cases, where
%% the english logo is used, the logo incorporates also the english
%% translation "Clausthal University of Technology" below the usual TU
%% Clausthal line.  Thus the height of the english logo is greater
%% than 24 mm.  
%% 
%% The width of the grey box at the right hand side of the paper
%% varies a bit.  Usually it is half as wide, as the square logo, thus
%% normally 12 mm wide.  Despite this rule, on a letterhead, the
%% various sender informations are put into that grey box, therefore
%% the box has to be wider than usual.  The CD Manual gives 48 mm as
%% total width, including a margin of 4 mm on each side. 
\newlength{\greyboxwidth}\setlength{\greyboxwidth}{12mm}
% \newlength{\titleskip}
% \setlength{\titleskip}{28mm}

%% Usually, you should use preprinted papers, which already have the
%% logo and the grey box of appropriate sizes filled in.  Be careful,
%% as there are at least two different preprinted papers available:
%% one for title pages of thesises and one for letters.
%% 
%% If you can not use those preprinted papers, we have to incorporate
%% the logo and grey box ourselves.  Fortunately, KOMA-Script classes
%% provide a lot of support, to print such delicate things.  Thank you
%% Markus.  In case, you view this kind of documents only in a digital
%% version (aka computer screen, beamer, ...), we should place the
%% logo and box directly at the edge of the paper.  But if you lack
%% the preprinted paper and have to print the logo and box yourself,
%% they should be separated from the edges by an amout of white space,
%% which depends on the respective printer.  Therefore, this file
%% provides two variables, to adapt the white space accordingly.

%% ................................................... &Titlepage-Layer ... 
%% 
%% The titlepages (either of a thesis or a letter) will be printed by
%% a KOMA script layer.  
\DeclareNewLayer[%
  align=tr,%
  background,%
  rightmargin,%
  contents={\hfill\textcolor{TUCgrey2}{\rule{\greyboxwidth}{\paperheight}}}]%
{TUCtitlepagelayer}%
\DeclareNewPageStyleByLayers{TUCtitlepagestyle}{TUCtitlepagelayer}%

%% .................................................. &Logo-Positioning ...
%% 
%% In order, to position the logo of Clausthal University of
%% Technology, we do need some variables.  \logovskip and \logohskip
%% represent the amount to offset the logo from the upper left corner
%% of the text area.
%% 
%% the new defined length \@logovcorr and \@logohcorr are used, to
%% offset the logo, in case the printer is not able to print that
%% close to the paper edges (which is almost allways the case).
\newlength{\@logovcorr}\setlength{\@logovcorr}{0pt}%
\newlength{\@logohcorr}\setlength{\@logohcorr}{0pt}%
%% 
%% The commands \logovcorr and \logohcorr are used, to set the the
%% offset by the user.  This command will only take effect, if the
%% option preprint was not set!
\newcommand{\logovcorr}[1]{%
  \if@preprint%
     \typeout{\MessageBreak Correction of the Logo alignment is not possible on preprinted papers! \MessageBreak}%
  \else%
     \setlength{\@logovcorr}{#1}%
    \fi}%
\newcommand{\logohcorr}[1]{%
  \if@preprint%
     \typeout{\MessageBreak Correction of the Logo alignment is not possible on preprinted papers! \MessageBreak}%
  \else%
     \setlength{\@logohcorr}{#1}%
  \fi}%
%% 
%% \logovskip and \logohskip represent the amount of white space to
%% place the Logo accordingly
%% TODO
\newlength{\logovskip}\newlength{\logohskip}%
\if@noindent%
  \setlength{\logovskip}{2.25\parskip}%
  \setlength{\logohskip}{0pt}%
\else%
  \setlength{\logovskip}{0.16in}%
  \setlength{\logohskip}{0.16in}%
\fi%
\addtolength{\logovskip}{1in}\addtolength{\logovskip}{\voffset}%
\addtolength{\logovskip}{\topmargin}\addtolength{\logovskip}{\headheight}\addtolength{\logovskip}{\headsep}%
\addtolength{\logovskip}{-\@logovcorr}%
\addtolength{\logohskip}{1in}\addtolength{\logohskip}{\hoffset}%
\addtolength{\logohskip}{\oddsidemargin}%
\addtolength{\logohskip}{-\@logohcorr}%
%% 
%% \logowidth declares the length of the Logo.  It depends on the
%% papersize.  While defined for DIN A4 papersize, the length of 130
%% mm will also be used on letter papersize.  On A5 paper, the length
%% will be 92 mm.
\newlength{\logowidth}%
\setlength{\logowidth}{130mm}%
\if@afivepaper%
   \setlength{\logowidth}{92mm}%
\fi%

%% ............................................................ &Tables ...
%% 
%% A tabular environment is needed so often in our documents, it seems
%% to be adequat, to define some table heads here, to use them
%% commonly.
%% 
%% Column for table heads.  Use a small fontsize and place the column
%% description centered.
\newcolumntype{H}{>{\footnotesize}c}%
%% 
%% New columns, to typeset columns of definded width with ragged
%% texts.  Argument #1 denotes the width of the column.  Keep in mind,
%% that the package ragged2e was load with option "newcommands" so
%% that indeed the new commands are used.
\newcolumntype{L}[1]{%
  >{\raggedright\arraybackslash\hspace{0pt}}p{#1}}%
\newcolumntype{C}[1]{%
  >{\centering\arraybackslash\hspace{0pt}}p{#1}}%
\newcolumntype{R}[1]{%
  >{\raggedleft\arraybackslash\hspace{0pt}}p{#1}}%
%% 
%% Decimal columns for Financial data or floating numbers.  You can
%% choose the number of decimal places by means of #1.  The input data 
%% of this F column (with capital letter F) uses the decimal point "."
%% to denote the decimals. The output uses a comma "," instead.
\newcolumntype{F}[1]{D{.}{,}{#1}}
%% Same as the F column, but this time, the decimal separator "," is
%% used for input and output.
\newcolumntype{f}[1]{D{,}{,}{#1}}

%% ............................................................. &Logos ...

%%  . . . . . . . . . . . . . . . . . . . . . . . . . . . . . &TUC-Logo . .
%% Define the colors, which are described in the CD Manual.  The
%% definition depends on the color scheme to be used.
\if@rgb%
  \typeout{^^J tuc-cd.sty: INFO: RGB in use!^^J } 
  \if@english%
    \def\@logofile{Logo/Logo-TUC-EN-RGB}%
  \else%
    \def\@logofile{Logo/Logo-TUC-DE-RGB}%
  \fi%
\else%
  \typeout{^^J tuc-cd.sty: INFO: RGB not in use!^^J } 
\fi%
\if@cmyk%
  \typeout{^^J tuc-cd.sty: INFO: CMYK in use!^^J } 
  \if@english%
    \def\@logofile{Logo/Logo-TUC-EN-CMYK}%
  \else%
    \def\@logofile{Logo/Logo-TUC-DE-CMYK}%
  \fi%
\fi%
\if@black%
  \typeout{^^J tuc-cd.sty: INFO: BLACK in use!^^J } 
  \if@english%
    \def\@logofile{Logo/Logo-TUC-EN-BLACK}%
  \else%
    \def\@logofile{Logo/Logo-TUC-DE-BLACK}%
  \fi%
\fi%


%%  . . . . . . . . . . . . . . . . . . . . . . . . . . . . &LaTeX-Logo . .
%% 
%% This command will present a much nicer LaTeX logo.  So let's use
%% that.  It was copied from an Article by Grzegorz Murzynowski
%% https://www.tug.org/TUGboat/tb29-1/tb91murzynowski-logo.pdf
\DeclareRobustCommand{\LaTeX}{%
  {%
    L%
    \setbox\z@\hbox{\check@mathfonts
      \fontsize\sf@size\z@
      \math@fontsfalse\selectfont
      A}%
    \kern-.57\wd\z@
    \sbox\tw@ T%
    \vbox to\ht\tw@{\copy\z@ \vss}% 
    \kern-.2\wd\z@}%
  {%
   \ifdim\fontdimen1\font=\z@
   \else
   \count\z@=\fontdimen5\font 
   \multiply\count\z@ by 64\relax 
   \divide\count\z@ by\p@ \count\tw@=\fontdimen1\font 
   \multiply\count\tw@ by\count\z@ 
   \divide\count\tw@ by 64\relax 
   \divide\count\tw@ by\tw@ 
   \kern-\the\count\tw@ sp\relax
   \fi}%
 \TeX}%


%% . . . . . . . . . . . . . . . . . . . . . . . . . . . . .  &TUC-plus . . 
%% This command prints the "Logo" of the TUC-plus project.
\newcommand{\tucplus}{\protect{TUC$^\mathsf{plus}$}}



%% ............................................................ &Layout ...
%% . . . . . . . . . . . . . . . . . . . . . . . . . . . . . . &Indents . .
\if@noindent
  \setlength{\parindent}{0pt}%
  \setlength{\parskip}{2ex plus0.5ex minus0.3ex}%
\else
  \setlength{\parindent}{2em}%
\fi


%% . . . . . . . . . . . . . . . . . . . . . . . . . . . . . . &itemize . .
%% Itemized environments should have a \circ in the second hierarchy.
% \renewcommand{\labelitemi}{$\bullet$}
% \renewcommand{\labelitemii}{$\Box$}
% \renewcommand{\labelitemiii}{$\circ$}
% \renewcommand{\labelitemiv}{$\triangleright$}



%%% ============================================================= &EOF ===

%%% Local Variables:
%%% mode: LaTeX
%%% TeX-engine: luatex 
%%% TeX-master: "~/lib/texmf/doc/latex/TUC/tuc-thesis.tex"
%%% TeX-parse-self: t
%%% TeX-auto-save: t
%%% End:

\include{typographie}
\include{latex-klassen}


%%% ----------------------------------------------------- &Back-Matter ---
\backmatter

%% Normally, this is the appendix
\printbibliography




\end{document}
%%% ============================================================= &EOF ===

%%% Local Variables:
%%% mode: LaTeX
%%% TeX-engine: luatex
%%% TeX-PDF-mode: t
%%% TeX-master: t
%%% TeX-parse-self: t
%%% TeX-auto-save: t
%%% End:

